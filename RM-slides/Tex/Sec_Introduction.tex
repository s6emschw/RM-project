%---------------------------------------------------------------------------%
\lecture{Research Presentation}{lec_present_intro}
%---------------------------------------------------------------------------%
\section{Introduction}%
%---------------------------------------------------------------------------%
\begin{frame}[fragile]
    \frametitle{Introduction: Drawbacks to OLS}
    \begin{itemize}
        \item Consider the following multivariate linear regression, where $Y=X\beta + \varepsilon$: 
            \begin{itemize}
                \item $Y=(y_1,...,y_n)$ $\in$ ${\rm I\!R^{n}}$ is the response vector
                \item $X$ $\in$ ${\rm I\!R^{n \times p}}$ is the predictor variable matrix
                \item $\varepsilon=¨(\varepsilon_1,...,\varepsilon_n)$ is the error vector
            \end{itemize}
        \item When is OLS a sub-optimal model for prediction? 
            \begin{itemize}
                \item Imperfect multicollinearity
                    \begin{itemize}
                        \item $cov(X_i, X_j)\neq 0$
                        \item $\beta_{OLS}$ still BLUE, but variance and standard errors will increase.
                        \item Leads to lower t-statistics.
                    \end{itemize}
                \item High-dimensionality ($p > n$)
                    \begin{itemize}
                        \item Rank $(X)$ $<$ $p$ such that OLS does not generate a unique solution.
                        \item Variance of estimated coefficients becomes infinitely large.
                    \end{itemize}
            \end{itemize}
    \end{itemize}
\end{frame}
%---------------------------------------------------------------------------%
\begin{frame}[fragile, c]
\frametitle{Regularization Methods}
    \begin{center}
        \textbf{Solution:} Use regularization method(s) to reduce the variance of the model fit in exchange for a marginal increase in the bias.
    \end{center}
\end{frame}
